\documentclass[12pt]{article}

\usepackage[T1]{fontenc}
\usepackage{bookmark}
\usepackage{amsmath}
\usepackage{graphicx}
\usepackage{hyperref}
\usepackage{titlesec}
\usepackage{tocloft}


\title{Projekt Indywidualny, Systemy Cyfrowe i Komputerowe
\\

\large Projekt wykonawczy modułu \textbf{sync\_arith\_unit\_12} 
operującej na liczbach w kodzie \textbf{ZM} }
\author{
  \begin{tabular}{c}
    Wykonująca: Anna Dzieżyk \\
    Opiekun: Bartosz Dec
  \end{tabular}
}
\date{Semestr zima 2023/2024}
\renewcommand{\cftsecleader}{\cftdotfill{\cftdotsep}}
\renewcommand{\cftsubsecleader}{\cftdotfill{\cftdotsep}}
\renewcommand{\cftsubsubsecleader}{\cftdotfill{\cftdotsep}}
\renewcommand{\contentsname}{Spis treści}

\begin{document}

\maketitle

\tableofcontents

\section{Cel i opis projektu}

Celem projektu jest implementacja synchronicznej
jednostki arytmetyczno-logicznej 
\textbf{sync\_arith\_unit\_12} 
realizującej operacje arytmetyczne, logiczne i inne zapisane na liczbach całkowitych zapisanych w kodzie 
\textbf{ZM}.

\setlength{\indent}{}

W skład realizacji projektu wchodzi:

\begin{itemize}
  \item Model układu cyfrowego opisany za pomocą języka opisu sprzętu 
  \textit{SystemVerilog}
  \item Synteza układu przy użyciu narzędzia open-source 
  \textit{yosys}
  \item Realizacja modułu(-ów) testowych 
  \textit{testbench} w celu weryfikacji poprawności działania zaimplementowanego modelu wraz z zamiejszczonymi 
  w niej raportami zawierającymi wyniki działania przed i po syntezie, jak również dane statystyczne
  dotyczące syntezy (dostarczone przez program 
  \textit{yosys}).
\end{itemize}

\section{Realizowane przez układ operacje}
Implementowany model ma zostać zaprojektowany tak, aby realizował poniżej
zapisane operacje na dwóch 
\textit{m-bitowych} wektorch wejściowych
\textbf{A} i \textbf{B}:
\begin{itemize}
  \item \textbf\and\textit{A} $>>$ \textbf\and\textit{\raisebox{-0.5ex}{\textasciitilde} B}
   
  Przesunięcie wektora \textbf\and\textit{A} o \textbf\and\textit{\raisebox{-0.5ex}{\textasciitilde} B}
  bitów w prawo. Jeżeli \textbf\and\textit{\raisebox{-0.5ex}{\textasciitilde} B} jest mniejsze
  od zera, układ ma zgłosiś błąd, a wartość wyjściowa ma zostać nieokreślona.
  \item \textbf\and\textit{A} $>$ \textbf\and\textit{\raisebox{-0.5ex}{\textasciitilde} B}
  
  Sprawdzenie, czy zaprzecona liczba \textbf{B} jest mniejsza od liczby \textbf{A}.
  Gdy warunek jest spełniony, układ ma wystawić na wyjściu liczbę dodatnią, w przeciwnym
  wypadku ma to być liczba równa 0.

  \item \raisebox{-0.5ex}{\textasciitilde} A[B] = 1
  
  Wynikiem operacji jest ustawienie bitu w liczbie \textbf\and\textit{\raisebox{-0.5ex}{\textasciitilde} A}
  o index \textbf\and\textit{B} (licząc od bitu najmniej znaczącego) ustawionym na wartość 1.
  Jeżeli liczba \textbf\and\textit{B} jest mniejsza od 0 lub większa od szerokości wektora \textbf\and\textit{A},
  układ ma zgłaszać błąd.

  
  \item \textbf\and\textit{ZM(A)=> U2(A)}  
  
  Zmiana liczby \textbf{A} zapisanej w kodzie ZM na zapis w kodzie U2. Jeżeli nie można
  dokonać poprawnej konwersji należy zgłosić błąd, a wyjście układu ma pozostać nieokreślone.

\end{itemize}

\section{Porty układu}

Układ ma mieć określone porty wejściowe i wyjściowe.
\begin{itemize}
  \item \textbf{i\textunderscore op} - n-bitowe wejście określające kod operacji
  \item \textbf{i\textunderscore arg\textunderscore A} - m-bitowe wejście argumentu A
  \item \textbf{i\textunderscore arg\textunderscore B} - m-bitowe wejście argumentu B
  \item \textbf{i\textunderscore clk} - wejście zegarowe układu
  \item \textbf{i\textunderscore reset} - wejście resetu synchronicznego wyzwalanego stanem niskim
  \item \textbf{o\textunderscore result} - wyjście synchroniczne z układu
\end{itemize}
Dodatkowo, jednostka arytmetyczna posiada 4-bitowe wyjście   synchroniczne \textbf{o\textunderscore status}
informujące określonymi bitami o statusie powiązanym z wynikiem operacji:

\begin{itemize}
  \item \textbf{bit ERROR} - sygnalizacja o tym, iż wynik operacji został określony niepoprawnie.
  \item \textbf{bit NOT\textunderscore  EVEN\textunderscore 0} - sygnalizuje nieparzystą liczbę zer w wyniku.
  Bit ten ma być ustawiany na 0 zawsze, gdy jest sygnalizowany błąd operacji.
  \item \textbf{bit ZEROS} - sygnalizuje, że wszystkie bity wyniku ustawione są na 0. Bit ten ma być
  ustawiony na 0 zawsze, gdy sygnalizowany jest błąd operacji.
  \item \textbf{bit OVERFLOW} - bit ten sygnalizuje, że nastąpiło przepełnienie i wynik 
  operacji wykracza poza szerokość wektora wyjściowego.
\end{itemize}

\section{Sprawozdanie z syntezy i realizacji jednostki 
\\ \textbf{sync\_arith\_unit\_12} }

\subsection{}
\end{document}
