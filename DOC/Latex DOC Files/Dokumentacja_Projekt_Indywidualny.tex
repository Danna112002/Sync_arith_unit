\documentclass[12pt]{article}

\usepackage[T1]{fontenc}
\usepackage{bookmark}
\usepackage{amsmath}
\usepackage{graphicx}
\usepackage{hyperref}
\usepackage{titlesec}
\usepackage{tocloft}


\title{Projekt Indywidualny, Systemy Cyfrowe i Komputerowe
\\

\large Projekt wykonawczy modułu \textbf{sync\_arith\_unit\_12} 
operującej na liczbach w kodzie \textbf{ZM} }
\author{
  \begin{tabular}{c}
    Wykonująca: Anna Dzieżyk \\
    Opiekun: Bartosz Dec
  \end{tabular}
}
\date{Semestr zima 2023/2024}
\renewcommand{\cftsecleader}{\cftdotfill{\cftdotsep}}
\renewcommand{\cftsubsecleader}{\cftdotfill{\cftdotsep}}
\renewcommand{\cftsubsubsecleader}{\cftdotfill{\cftdotsep}}
\renewcommand{\contentsname}{Spis treści}

\begin{document}

\maketitle

\tableofcontents

\section{Cel i opis projektu}

Celem projektu jest implementacja synchronicznej
jednostki arytmetyczno-logicznej 
\textbf{sync\_arith\_unit\_12} 
realizującej operacje arytmetyczne, logiczne i inne zapisane na liczbach całkowitych zapisanych w kodzie 
\textbf{ZM}.

\setlength{\indent}{}

W skład realizacji projektu wchodzi:

\begin{itemize}
  \item Model układu cyfrowego opisany za pomocą języka opisu sprzętu 
  \textit{SystemVerilog}
  \item Synteza układu przy użyciu narzędzia open-source 
  \textit{yosys}
  \item Realizacja modułu(-ów) testowych 
  \textit{testbench} w celu weryfikacji poprawności działania zaimplementowanego modelu wraz z zamiejszczonymi 
  w niej raportami zawierającymi wyniki działania przed i po syntezie, jak również dane statystyczne
  dotyczące syntezy (dostarczone przez program 
  \textit{yosys}).
\end{itemize}

\section{Realizowane przez układ operacje}
Implementowany model ma zostać zaprojektowany tak, aby realizował poniżej
zapisane operacje na dwóch 
\textit{m-bitowych} wektorch wejściowych
\textbf{A} i \textbf{B}:
\begin{itemize}
  \item \textbf\and\textit{A} $>>$ \textbf\and\textit{\raisebox{-0.5ex}{\textasciitilde} B}
   
  Przesunięcie wektora \textbf\and\textit{A} o \textbf\and\textit{\raisebox{-0.5ex}{\textasciitilde} B}
  bitów w prawo. Jeżeli \textbf\and\textit{\raisebox{-0.5ex}{\textasciitilde} B} jest mniejsze
  od zera, układ ma zgłosiś błąd, a wartość wyjściowa ma zostać nieokreślona.
  \item \textbf\and\textit{A} $>$ \textbf\and\textit{\raisebox{-0.5ex}{\textasciitilde} B}
  
  Sprawdzenie, czy zaprzecona liczba \textbf{B} jest mniejsza od liczby \textbf{A}.
  Gdy warunek jest spełniony, układ ma wystawić na wyjściu liczbę dodatnią, w przeciwnym
  wypadku ma to być liczba równa 0.

  \item \raisebox{-0.5ex}{\textasciitilde} A[B] = 1
  
  Wynikiem operacji jest ustawienie bitu w liczbie \textbf\and\textit{\raisebox{-0.5ex}{\textasciitilde} A}
  o index \textbf\and\textit{B} (licząc od bitu najmniej znaczącego) ustawionym na wartość 1.
  Jeżeli liczba \textbf\and\textit{B} jest mniejsza od 0 lub większa od szerokości wektora \textbf\and\textit{A},
  układ ma zgłaszać błąd.

  
  \item \textbf\and\textit{ZM(A)=> U2(A)}  
  
  Zmiana liczby \textbf{A} zapisanej w kodzie ZM na zapis w kodzie U2. Jeżeli nie można
  dokonać poprawnej konwersji należy zgłosić błąd, a wyjście układu ma pozostać nieokreślone.

\end{itemize}

\section{Porty układu}

Układ ma mieć określone porty wejściowe i wyjściowe.
\begin{itemize}
  \item \textbf{i\textunderscore op} - n-bitowe wejście określające kod operacji
  \item \textbf{i\textunderscore arg\textunderscore A} - m-bitowe wejście argumentu A
  \item \textbf{i\textunderscore arg\textunderscore B} - m-bitowe wejście argumentu B
  \item \textbf{i\textunderscore clk} - wejście zegarowe układu
  \item \textbf{i\textunderscore reset} - wejście resetu synchronicznego wyzwalanego stanem niskim
  \item \textbf{o\textunderscore result} - wyjście synchroniczne z układu
\end{itemize}
Dodatkowo, jednostka arytmetyczna posiada 4-bitowe wyjście   synchroniczne \textbf{o\textunderscore status}
informujące określonymi bitami o statusie powiązanym z wynikiem operacji:

\begin{itemize}
  \item \textbf{bit ERROR} - sygnalizacja o tym, iż wynik operacji został określony niepoprawnie.
  \item \textbf{bit NOT\textunderscore  EVEN\textunderscore 0} - sygnalizuje nieparzystą liczbę zer w wyniku.
  Bit ten ma być ustawiany na 0 zawsze, gdy jest sygnalizowany błąd operacji.
  \item \textbf{bit ZEROS} - sygnalizuje, że wszystkie bity wyniku ustawione są na 0. Bit ten ma być
  ustawiony na 0 zawsze, gdy sygnalizowany jest błąd operacji.
  \item \textbf{bit OVERFLOW} - bit ten sygnalizuje, że nastąpiło przepełnienie i wynik 
  operacji wykracza poza szerokość wektora wyjściowego.
\end{itemize}

\section{Sprawozdanie z syntezy i realizacji jednostki 
\\ \textbf{sync\_arith\_unit\_12} }


Realizacja projektu polegała na zaprojektowaniu czterech małych modułów, 
zsyntezowaniu ich, po czym przetestowaniu ich samodzielnej pracy w module testbenchu.

Następnie po potwierdzeniu poprawności pracy w.w. modułów, dołączyłam ich instancje
do pliku z modułem ALU. 
Moduł ALU łączy działanie wszystkich czterech modułów. Jego funkcjonaność polega na
przyjęciu wartości\footnote[1]{Lista portów układu znajduje się w punkcie 3. Listy portów do poszczególnych modułów
znajdą się w ich opisach poniżej.}:
\begin{itemize}
  \item arumentów A, B, 
  \item dwubitowego operandu wyboru operacji, 
  \item sterowanego resetu synchronicznego,
  \item zegara.
\end{itemize}
Wtedy wykonywane jest działanie, a następnie generowane są wartości wyjść: wyniku i statusu operacji.
Do tego modułu też napisałam testbench sprawdzający działanie jednostki arytmetyczno-logicznej.
Wszystkie testy - i te jednostkowe i testy całości wykazały poprawne zachowanie układu i oczekiwane wyniki. 

\subsection{Moduł przesunięcia bitowego (\textbf\and\textit{A} $>>$ \textbf\and\textit{\raisebox{-0.5ex}{\textasciitilde} B})}
Działanie tego modułu zostało opisane w podpunkcie 1. punktu 2. \footnote[2]{Działanie każdego "małego" modułu opisane jest w punkcie 2., 
dlatego w pozostałych raportach z syntez małych modułów nie będzie opisów ich działania} 


\textbf{Lista portów modułu}
\begin{itemize}
  \item \textbf{i\textunderscore arg\textunderscore A, i\textunderscore arg\textunderscore B} - wejścia zapewniające 
  operandy, na których będzie wykonywana operacja
  \item \textbf{o\textunderscore result} - wynik operacji
  \item \textbf{o\textunderscore error} - flaga przyjmująca wartość 1, gdy wykrywany jest błąd operacji
\end{itemize}

\textbf{Symulacja i synteza}

Synteza modułu przesunięcia przebiegła bez zatrzasków, w pliku synth.log nie znajdujemy też błędów. Po syntezie, działanie
obydwu plików z modułami przed syntezą ("przesuniecie.sv") i po ("przesuniecie\textunderscore rtl.sv") zostało sprawdzone w module "testbench".
Wyniki przebiegów zegarowych tej symulacji możemy obserwować w programie GTKWave, co obrazuje poniższy rysunek.

\begin{figure}[h]
\centering
\includegraphics[scale=.5]{/home/ania/Obrazy/Zrzuty ekranu/Zrzut ekranu z 2023-12-19 05-34-29.png}
\centering
\caption{\textit{Przebiegi sygnałów użytych w testbenchu, pokazane w programie GTKWave}}
\end{figure}

Stwierdzenie poprawności zachowania modułu "przesunięcie.sv" umożliwiła konktretna konstrukcja testbenchu.
Pierwsze 5 wartości operandów A oraz B, czyli przebieg do szóstej sekundy, miał obrazować sytuację "idealną", 
która miała spełniać wymogi: 
\begin{itemize}
\item A wylosowane z przedziału liczb możliwych do zapisania na m-bitach,
\item B wylosowane z przedziału liczb od 0 do maksymalnej liczby bitów A.
\end{itemize}
Więc: flaga błędu nie jest podniesiona, wyniki modułów przed syntezą i po są identyczne.

Przypadek ten powtarza się dla przesunięć przyjmujących wartości "dodatnie zero" oraz "ujemne zero", występujących
w kodowaniu liczb ZM (od 6. do 16. sekundy), dla przesunięcia większego od 32 bitów oraz dla szczególnych przypadków, gdzie operandy 
A i B są zerami "dodatnimi" i "ujemnymi".

Jedyna sytuacja, w której podnoszona jest flaga błędu to od 16. do 21. sekundy, wtedy też w module przed syntezą wynik 
jest niezdefiniowany.

\newpage
\subsection{Moduł porównania liczb (\textbf\and\textit{A} $>$ \textbf\and\textit{\raisebox{-0.5ex}{\textasciitilde} B})}

\textbf{Lista portów modułu}
\begin{itemize}
  \item \textbf{i\textunderscore arg\textunderscore A, i\textunderscore arg\textunderscore B} - wejścia zapewniające 
  operandy, na których będzie wykonywana operacja
  \item \textbf{o\textunderscore result} - wynik operacji
\end{itemize}

\textbf{Symulacja i synteza}

Synteza modułu porównania przebiegła bez zatrzasków, w pliku synth.log nie znajdujemy też błędów. Po syntezie, działanie
obydwu plików z modułami przed syntezą ("porownanie.sv") i po ("porownanie\textunderscore rtl.sv") zostało sprawdzone w module "testbench".
Wyniki przebiegów zegarowych tej symulacji możemy obserwować w programie GTKWave, co obrazuje poniższy rysunek.

\begin{figure}[h]
  \centering
  \includegraphics[scale=.45]{/home/ania/Obrazy/Zrzuty ekranu/Zrzut ekranu z 2023-12-19 06-45-18.png}
  \centering
  \caption{\textit{Przebiegi sygnałów użytych w testbenchu, pokazane w programie GTKWave}}
  \end{figure}

Więc: do 10. sekundy obserówjemy porównywanie losowych wartości A i B, następnie aż do końca symulacji
A i B przyjmują wartości zer "dodatnich" i "ujemnych" mając dawać wynik porównania równy 0.
Po wyrywkowym sprawdzeniu pierwszej części testbenchu oraz drugiej, można stwierdzić, że moduł działa prawidłowo.
\subsection{Moduł ustawiania wartości 1 na zadanym bicie 
\\(\raisebox{-0.5ex}{\textasciitilde} A[B] = 1)}

\textbf{Lista portów modułu}
\begin{itemize}
  \item \textbf{i\textunderscore arg\textunderscore A, i\textunderscore arg\textunderscore B} - wejścia zapewniające 
  operandy, na których będzie wykonywana operacja
  \item \textbf{o\textunderscore result} - wynik operacji
  \item \textbf{o\textunderscore error} - flaga przyjmująca wartość 1, gdy wykrywany jest błąd operacji
\end{itemize}

\textbf{Symulacja i synteza}

Synteza modułu ustawienia bitu na 1 przebiegła bez zatrzasków, w pliku synth.log nie znajdujemy też błędów. Po syntezie, działanie
obydwu plików z modułami przed syntezą ("ustawienie.sv") i po ("ustawienie\textunderscore rtl.sv") zostało sprawdzone w module "testbench".
Wyniki przebiegów zegarowych tej symulacji możemy obserwować w programie GTKWave, co obrazuje poniższy rysunek.

\begin{figure}[h]
  \centering
  \includegraphics[scale=.45]{/home/ania/Obrazy/Zrzuty ekranu/Zrzut ekranu z 2023-12-19 07-29-16.png}
  \centering
  \caption{\textit{Przebiegi sygnałów użytych w testbenchu, pokazane w programie GTKWave}}
  \end{figure}

  Budowa testbencha umożliwia na poprawne wykonanie operacji ustawienia bitu od 1. do 7. sekundy. Następnie sprawdzane 
  jest podnoszenie flagi error dla B mniejszego od zera i większego od zakresu bitów A. Ostatnie dwie sekundy symulacji są poświęcone
  na sprawdzenie poprawności zachowania modułu dla dwóch różnych reprezentacji zera w argumencie B - dla nich nie powinna się 
  pojawiać flaga error. Na podstawie powyższych faktów można uznać, że moduł ustawiania bitu na 1 działa poprawnie.

\subsection{Moduł konwersji zapisu liczby z ZM na U2 
\\(\textbf\and\textit{ZM(A)=> U2(A)})}

\textbf{Lista portów modułu}
\begin{itemize}
  \item \textbf{i\textunderscore arg\textunderscore A} - wejście zapewniające 
  operand, na którym będzie wykonywana operacja
  \item \textbf{o\textunderscore result} - wynik operacji
  \item \textbf{o\textunderscore error} - flaga przyjmująca wartość 1, gdy wykrywany jest błąd operacji
\end{itemize}

\textbf{Symulacja i synteza}

Synteza modułu konwersji przebiegła bez zatrzasków, w pliku synth.log nie znajdujemy też błędów. Po syntezie, działanie
obydwu plików z modułami przed syntezą ("konwersja.sv") i po ("konwersja\textunderscore rtl.sv") zostało sprawdzone w module "testbench".
Wyniki przebiegów zegarowych tej symulacji możemy obserwować w programie GTKWave, co obrazuje poniższy rysunek.

\begin{figure}[h]
  \centering
  \includegraphics[scale=.5]{/home/ania/Obrazy/Zrzuty ekranu/Zrzut ekranu z 2023-12-19 14-24-58.png}
  \centering
  \caption{\textit{Przebiegi sygnałów użytych w testbenchu, pokazane w programie GTKWave}}
  \end{figure}

  Testbench "konwersja\textunderscore tb.sv" napisany jest w taki sposób, by pierwsze 7 sekund moduł konwertował 
  liczby będące w zakresie konwersji, DUT wykonało poprawnie wszystkie te operacjem flaga error nie jest podniesiona.
  Następnie sprawdzany jest przypadek "ujemnego" zera, którego nie da się poprawnie przekonwertować, DUT ma wtedy podnieść 
  \textbf{o\textunderscore error},  wynik ma pozostać niezdefiniowany. Ostatnie 3 testy sprawdzają
  zachowanie modułu dla "dodatniego" zera oraz maksymalnej i minimalnej wartości z zakresu konwersji.
\subsection{Moduł ALU}

Moduł ALU połączył 4 moduły w całość oraz otrzymał kilka nowych sygnałów, nieużywanych w "małych modułach"
\textbf{Lista portów modułu}
\begin{itemize}
  \item \textbf{i\textunderscore arg\textunderscore A, i\textunderscore arg\textunderscore B} - wejście zapewniające 
  operandy, na których będzie wykonywana operacja
  \item \textbf{o\textunderscore result} - wynik operacji
  \item \textbf{o\textunderscore error} - flaga przyjmująca wartość 1, gdy wykrywany jest błąd operacji. Każdej operacji, w której możliwe
  jest wykrycie błędu, odpowiada wewnętrzny oddzielnie nazwany i zadeklarowany sygnał błędu.
  \item \textbf{o\textunderscore status} - synchroniczne wyjście z ALU pozwalające sprawdzić: czy wynik posiada parzystą liczbę zer, 
  czy wynik składa się z samych zer, czy wynik jest błędny oraz czy nastąpiło przepełnienie\footnote[3]{Przepełnienie nie jest możliwe do pojawienia się
  w żadnym z "małych" modułów.} Wyjście to ma szerokość czterech bitów, ALU sprawdza zaistnienie w.w. sytuacji przy wykonywaniu operacji i podnosi odpowiednie
  flagi.
  \item \textbf{i\textunderscore clk} - wejście sygnału zegarowego.
  \item \textbf{i\textunderscore reset} - wejście resetu synchronicznego, aktywowanego narastającym zboczem zegara.
  \item \textbf{i\textunderscore op} - wejście ustalające, jaka operacja zostanie wykonana.
\end{itemize}

\textbf{Symulacja i synteza}

Synteza "sync\textunderscore arith\textunderscore unit\textunderscore 12.sv" przebiegła bez zatrzasków, w pliku synth.log nie znajdujemy też błędów. Po syntezie, działanie
obydwu plików z modułami przed syntezą ("sync\textunderscore arith\textunderscore unit\textunderscore 12.sv") i po ("sync\textunderscore arith\textunderscore unit\textunderscore 12\textunderscore rtl.sv") 
zostało sprawdzone w module "testbench".
Wyniki fragmentu przebiegów zegarowych tej symulacji możemy obserwować w programie GTKWave, co obrazuje poniższy rysunek.

\begin{figure}[h]
  \centering
  \includegraphics[scale=.5]{/home/ania/Obrazy/Zrzuty ekranu/Zrzut ekranu z 2023-12-19 15-20-32.png}
  \centering
  \caption{\textit{Przebiegi sygnałów użytych w testbenchu, pokazane w programie GTKWave}}
  \end{figure}

  Dane modelu (ALU przed syntezą) i bramek (ALU po syntezie) są zgodne wszędzie tam, gdzie wynik nie pozostaje niezdefiniowany. Wynik pozostaje niezdefiniowany tylko w sytuacjach, 
  gdy jednostka ma wykonać przesunięcie bitowe (kod 11), a sygnał B reprezentuje wartość ujemną, czyli dokładnie tak, jak się można było spodziewać. Sygnał resetu został ustawiony 
  przeciwnie do zegara, by umożliwić zwalnianie danych z wyjścia przed otrzymaniem wyniku następnej operacji. Biorąc pod uwagę analizę przebiegu sygnałów, działanie
  jednostki jest poprawne.


\section{Schemat blokowy realizowanej jednostki \\\textbf{sync\_arith\_unit\_12}}

\begin{figure}[h]
  \centering
  \includegraphics[scale=.5]{/home/ania/Obrazy/Schemat blokowy ALU.jpg}
  \centering
  \caption{\textit{Schemat blokowy realizowanej jednostki wraz z nazwanymi w niej sygnałami}}
  \end{figure}

\end{document}
